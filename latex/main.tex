\documentclass{IEEEtran}

% For english labels:
\usepackage{EVjour}

% For slovene labels:
% \usepackage[slovene]{EVjour}

\usepackage{amssymb}
\usepackage{amsmath}
\usepackage{amsfonts}
\usepackage{color}
\usepackage{xspace}
\usepackage[pdftex]{graphicx}

\DeclareGraphicsExtensions{.pdf,.jpeg,.png}
\usepackage[hyphens]{url}
\usepackage[unicode]{hyperref}

\newcommand{\B}[1]{\ensuremath{\boldsymbol{#1}}}
\newcommand{\bi}[1]{\boldmath{\ensuremath{#1}}}
\newcommand{\CC}{\ensuremath{\mathbb C}}
\newcommand{\D}{\ensuremath{\mathbb D}}
\newcommand{\F}{\ensuremath{\mathbb F}}
\newcommand{\K}{\ensuremath{\mathbb K}}
\newcommand{\N}{\ensuremath{\mathbb N}}
\newcommand{\Q}{\ensuremath{\mathbb Q}}
\newcommand{\R}{\ensuremath{\mathbb R}}
\newcommand{\Z}{\ensuremath{\mathbb Z}}

\makeatletter
\let\old@subsection\subsection
\renewcommand{\subsection}[1]{\bigskip\old@subsection{#1}\@afterindentfalse\@afterheading}
\makeatother

% English labels
\newtheorem{theorem}{Theorem}[section]
\newtheorem{corollary}[theorem]{Corollary}
\newtheorem{lemma}[theorem]{Lemma}

% Slovene labels
% \newtheorem{theorem}{Izrek}[section]
% \newtheorem{corollary}[theorem]{Posledica}
% \newtheorem{lemma}[theorem]{Lema}


\pdfminorversion=4

\title{Project template}
\authors{Matjaž Pogačnik}
\address{
faculty of computer and information science \newline
večna pot 113\newline
1000 ljubljana}
\date{november 2020}
\abstract{
    Kratek opis ideje
}
\keywords{equiz, rating}

\begin{document}

\maketitle

\section{Introduction}
\label{sec:intro}

Opis problema

%\section{Attacks on Bitcoin}
%\label{sec:attacks}
\section{Glicko rating}
\label{sec:glicko}
Opis glicko ratinga, kot resitev zgornjega problema

%primer enacbe
\begin{align}
n &= pq \nonumber \\
\varphi(n) &= (p-1)(q-1) \label{eqn:phi}
\end{align}


\subsection{Izracuni iz equiz podatkov}

Trenutni podaki (lahko) baza za nove,
\\
\\
Po rating updatih lahko dolocimo cas (in verjetno kaj je ucenec reseval) -- match lahko uporabimo kar prvi rating in matchamo casovno naslednje preizkuse
\\
\\
Ali imamo podatke nakljucnih resevanj? (Nejc Subic diplomska: Da?)
Ce ne lahko tudi umetno prikazemo ucinek za sprotno delo -- nakljucno izberem nekaj nalog(lahko tudi randomly generiram podobne podatke kot trenutne naloge v equizu).
\\
\begin{itemize}
    \item Prikaz ratingov po izpitu (tudi za naloge). kot poseben primer za prikaz da je se vedno accurate -- izstopanje nalog
    \item Prikaz ratingov po rednih preverjanjih, kot prikaz kako se vzdrzuje rating in devianca(graf deviance za studenta) -- c samo pokaze koliko globoko bo dipnilo med preverjanji in s tem koliko pomembno je sprotno delo?
\end{itemize}
\hfill
\\
\\
\\
\\
\\
\\


Zeljeni podatki:
\begin{itemize}
    \item izpit (sasa)
    \item cas posodobitev ratingov + kaj je ucenec takrat reseval (za sestavljene naloge (izpite) + nakljucno resevanje)
\end{itemize}
\hfill
\\
\\
\begin{lemma} \label{lem:eps}
$\ell = \varepsilon + t(p-1)(q-1)$.
\end{lemma}

\begin{proof}
\begin{align*}
\ell &= \varepsilon + \B{r}(q-1) \\
&= \varepsilon + t(p-1)(q-1)
\end{align*}
\end{proof}

\begin{theorem} \label{thm:equiv}
$\ell \equiv \varepsilon \pmod{\varphi(n)}$.
\end{theorem}

\begin{proof}
By Lemma~\ref{lem:eps} and \eqref{eqn:phi} we have $\varepsilon \bmod{\varphi(n)} = \ell$.
\end{proof}

\begin{corollary} \label{cor:mult}
$\exists k \in \Z: \ell - \varepsilon = k \varphi(n)$.
\end{corollary}

\begin{proof}
Follows by Theorem~\ref{thm:equiv}.
\end{proof}

\section{Conclusion}
\label{sec:cnc}
Ut sed metus consectetur, vulputate lacus non, mollis libero~\ref{sec:glicko}.
In ut sollicitudin dolor \bi{n = pq}.
Sed condimentum nisl at tristique lacinia (Corollary~\ref{cor:mult}).
Ut laoreet orci sit amet eleifend aliquet.
Fusce purus massa, bibendum in felis eu, aliquam tincidunt est.
Phasellus quis malesuada quam.
Maecenas id felis est.
Nullam vulputate finibus augue, id ultricies nisi elementum a.

\bibliographystyle{babplain}
\bibliography{cite}

\end{document}

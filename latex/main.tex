\documentclass{IEEEtran}

% For english labels:
\usepackage{EVjour}

% For slovene labels:
% \usepackage[slovene]{EVjour}

\usepackage{amssymb}
\usepackage{amsmath}
\usepackage{amsfonts}
\usepackage{color}
\usepackage{xspace}
\usepackage[pdftex]{graphicx}

\DeclareGraphicsExtensions{.pdf,.jpeg,.png}
\graphicspath{ {./images/} }
\usepackage[hyphens]{url}
\usepackage[unicode]{hyperref}

\newcommand{\B}[1]{\ensuremath{\boldsymbol{#1}}}
\newcommand{\bi}[1]{\boldmath{\ensuremath{#1}}}
\newcommand{\CC}{\ensuremath{\mathbb C}}
\newcommand{\D}{\ensuremath{\mathbb D}}
\newcommand{\F}{\ensuremath{\mathbb F}}
\newcommand{\K}{\ensuremath{\mathbb K}}
\newcommand{\N}{\ensuremath{\mathbb N}}
\newcommand{\Q}{\ensuremath{\mathbb Q}}
\newcommand{\R}{\ensuremath{\mathbb R}}
\newcommand{\Z}{\ensuremath{\mathbb Z}}

\makeatletter
\let\old@subsection\subsection
\renewcommand{\subsection}[1]{\bigskip\old@subsection{#1}\@afterindentfalse\@afterheading}
\makeatother

% English labels
\newtheorem{theorem}{Theorem}[section]
\newtheorem{corollary}[theorem]{Corollary}
\newtheorem{lemma}[theorem]{Lemma}

% Slovene labels
% \newtheorem{theorem}{Izrek}[section]
% \newtheorem{corollary}[theorem]{Posledica}
% \newtheorem{lemma}[theorem]{Lema}


\pdfminorversion=4

\title{Project template}
\authors{Matjaž Pogačnik}
\address{
faculty of computer and information science \newline
večna pot 113\newline
1000 ljubljana}
\date{november 2020}
\abstract{
    Kratek opis ideje
}
\keywords{equiz, rating}

\begin{document}

\maketitle

\section{Introduction}
\label{sec:intro}
Ena izmed funkcionalnosti aplikacije eQuiz %TODO link
je rating študentov in nalog preko Elo rating sistema. Študent se sooči z nalogo in na ta način se iztočasno izračuna rating študenta in rating naloge, kot bi igrala študent in naloga šah. Študent lahko na nalogo odgovori prav ali napačno, kar je ekvivalentno temu, da štuent nalogo premaga ali proti njej izgubi. 
Pojavijo pa se pomankljivosti takega rating sistema. 
\hfill
\\
Če uporabljamo eQuiz za spremljanje ratinga študentov in nalog v sklopu določenega predmeta, bo proti številu študentov, ki bodo v tem letu aktivni, število nalog navadno mnogo večje. To pa pomeni, da bodo, dokler študentje ne rešujejo samih enakih nalog, ti prejeli posodobitev ratinga veliko večkrat kot kot posamezna naloga. Še več posodobitev pa študentje, ki so pri reševanju nalog bolj aktivni.
Torej ratigini, pripisani določeni nalogi ali študentu lahko izvirajo iz več ali manj reševanj, kar naredi rating bolj ali manj verodostojen.
Zaupanje v določen rating lahko tako iz posamezne številke raje razširimo na interval zaupanja.
\section{Glicko rating}
\label{sec:glicko}
Za tak pristop je primeren Glicko %TODO link
rating, ki poleg ratinga za posameznega igralca (študent ali naloga) vpeljuje še deviacijo $RD$, ki nam omogoča predstavitev ratinga posameznega igralca kot $95\%$ interval zaupanja:

\begin{equation}
    \left ( r -1.96RD, r +1.96RD \right )
\end{equation}
%lahko se primer izrazuna intervala
pri čemer se $RD$ manjša ob vsakem updatu ratinga, kjer se igralec sooči z drugim igralcem (ali več njih)
\begin{equation}
    RD'=\sqrt{\left (  \frac{1}{RD^{2}} + \frac{1}{d^{2}}\right )^{-1}}
\end{equation}
kjer $RD$ predstavlja deviacijo pred posodobitvijo, $d^{2}$ pa je definiran kot
\begin{equation}
    d^{2}=\left ( q^{2} \sum_{j=1}^{m} \left (g\left( RD_{j} \right ) \right )^{2}E\left ( s|r, r_{j}, RD_{j} \right )\left (1-E\left ( s|r, r_{j}, RD_{j} \right ) \right )\right )^{-1} 
\end{equation}
\begin{equation}
    q=\frac{ln10}{400}=0.0057565
\end{equation}
\begin{equation}
    E\left ( s|r, r_{j}, RD_{j} \right )=\frac{1}{1+10^{-g\left ( RD_{j} \right )\left ( r-r_{j} \right )/400}}
\end{equation}
\begin{equation}
g\left ( RD \right )=\frac{1}{\sqrt{1+3q^{2}\left ( RD^{2} \right )/\pi^{2} }}
\end{equation}
\hfill
\\
\\
Nov rating $r'$ se izračuna po formuli
\begin{equation}
    r'=r+\frac{q}{1/RD^{2}+1/d^{2}}\sum_{j=1}^{m}g\left( RD_{j} \right )\left (s_{j}-E\left ( s|r, r_{j}, RD_{j} \right ) \right )
\end{equation}
Zgornje formule so posplošene za posodobitev ratinga $r$ in deviacije $RD$ igralca proti skupini $m$ nasprotnikov z deviacijami 
$RD_{1}, RD_{2}, ..., RD_{m}$ in ratingi $r_{1}, r_{2}, ..., r_{m}$. $s_{1}, s_{2}, ..., s_{m}$ predstavljajo izide, ki so lahko $0$ ali $1$, za izgubo ali zmago.
\hfill
\\
\\
Igralčev $RD$ se ne posodablja samo, ko se ta sooča, temveč tudi ob preteku določene časovne periode, kar predstavlja zniževanje verodostojnosti trenutnega ratinga, če igralec določen čas ne igra. %TODO: poveži z equiz
Tako se njegov $RD$ posodobi kot
\begin{equation}
    RD = min\left(\sqrt{RD_{old}^{2} + c^{2}}, 350\right)
\end{equation}
konstanta $c$ je, poleg maksimalne deviacije, ki je v tem primeru $350$, edini parameter, ki ga nastavlja administrator sistema. Več o tem v kontekstu equiza v poglavju $X$ %TODO link do poglavja
\hfill
\\
Matematične izpeljave formul so na voljo na \href{http://www.glicko.net/research/glicko.pdf}{tukaj} %TODO organizacija formul v podnaslove?
\newpage
%TODO Podnaslov Intuitivno?
Iz formul vidimo, da je $r'$ odvisen od $RD$ igralca in ni uravnotežen tako, kot je pri Elo sistemu, kjer se zmagovlacu zviša rating toliko, kot se poražencu zniža. Ker naj bi velikost $RD$ odražala koliko informacije imamo o igralcu, se v iteraciji ocenjevanja, kjer ima en igralec velik $RD$, drug pa majhen, prvemu zviša veliko več, kot se drugemu zniža, saj vemo, da je rating prvega nezanesljiv, rating drugega pa je. Enako se v tem primeru ne more upravičeno znižati rating drugega igralca toliko, kot bi se, če bi bil rating njegovega soigralca zanesljiv.

\hfill
\\
%se to vkljuci
A useful way to summarize a player’s strength is through a confidence interval (or more particularly, given the quasi-Bayesian derivation of the Glicko system, a “credible” interval) rather than just reporting a rating. The confidence interval has the interpretation of re- porting the interval of plausible values for the player’s actual strength, acknowledging that a rating is merely an estimate of a player’s unknown true strength. A common choice is to report a $95\%$ confidence interval which provides $95\%$ confidence that the player’s true ability is within the interval. The formula for a $95\%$ confidence interval for a player with rating r and ratings deviation RD is given by

\hfill

\section{Izračuni iz eQuiz podatkov}
\label{sec:equiz}

\subsection{Pridobljeni podatki}
Iz podatkovne baze eQuiza so bili pridobljeni trenutni ratingi študentov in nalog ter kronološki zapis reševanj quizov, sestavljenih iz različnih nalog. Podatek o reševanju posamezne naloge v quizu je bil izračunan na podlagi sprememb ratingov nalog po končanem quizu. V tabelo podatkov je bil tako dodan stolpec correct z vrednostjo 0 ali 1 za vsako nalogo, kjer so naloge s podvprašanji razdeljene v posamezne naloge. Ta podatek je bil uporabljen tudi v kasnejšnjih izračunih ELO in Glicko ratingov.
\hfill
\\
Za izračun Glicko ratinga je, za razliko od ELO ratinga, potreben še čas reševanja. Zato so bile posodobitve ratingov združene s tabelo reševanj, ki vsebuje potrebne čase. Pri tem so se pojavila neskladja, kjer nekatere posodobitve ratingov, ki se nanašajo na nek identifikator kviza, niso bile nikdar zabeležene, nekatera zabeležena reševanja pa se ne nanašajo na noben rating, kar pomeni, da so k izračunu ratinga nekoč lahko prispevali različni tipi ocenjevanj, trenutno pa se v rating štejejo samo Rating Quizi. Tako so bili pridobljeni podatki za 305 različnih uporabnikov in 1348 nalog, za nadaljnje izračune pa so bili uporabni podatki za 261 uporabnikov in 1299 nalog.

Za izračune ratingov so bile od najstarejšega ratinga uporabnika po času reševanj zaporedoma pregledane naloge iz ustreznega kviza, ki so bile glede na prejšnji najstarejši ali (osnovni rating) uporabnika in naloge ponovno ratane po izbranem algoritmu. Tako je bil po zaporednih nalogah glede na reševanje (stolpec correct) izračunan nov rating uporabnika in novi ratingi nalog, ki so bili sproti zapisani v tabele na mesta z ustreznimi časi in identifikatorji izpitov, za uporabo v naslednjih korakih algoritma. Ko smo dosegli konec zabeleženih ratingov smo imeli izračunane ratinge za vse naloge in userje.

%TODO:povej da je bil potem vzet zadnji rating
%

\subsection{Trenutni rating sistem - ELO}
Pri oddaji kviza so pridobljeni prejšnji rating posameznih nalog v kvizu in prejšnji rating uporabnika. Za naloge in uporabnika so potem zaporedoma po nalogah izračunani novi ratingi po naslednjih enačbah:
\begin{equation}
    R'_{A}=R_{A}+K\cdot \left ( S_{A}-E_{A} \right )
\end{equation}
\begin{equation}
    E_{A}=\frac{Q_{A}}{Q_{A}+Q_{B}}
\end{equation}
Kjer so
\begin{equation}
    Q_{A}=10^{R_{A}/66}\;\;\;Q_{B}=10^{R_{B}/66}
\end{equation}
$S_{A}$ predstavlja dejanski izid, v primeru eQuiza 0 ali 1, $E_{A}$ pa pričakovan izid.
\hfill
\\
\\
$K$ je po navadi izračunan glede na število iger na katerih temelji trenuten rating (plus število iger v turnamentu pri šahu),
v trenutni implementaciji ratinga pa je privzet kot konstanten z vrednostjo 4,5
\hfill
\\
\\
Po takem postopku je bil po zgoraj opisanih algoritmih izračunan ELO rating na filtriranih podatkih z osnovnim ratingom 200.
%TODO normalne?
\begin{figure}[h]
\includegraphics[width=8cm]{ComputedELO}
\end{figure}
\begin{figure}[h]
\includegraphics[width=8cm]{ComputedELOe}
\end{figure}
Zaradi velikega števila nalog iz katerih naključno izbira kviz, so posamezne naloge povprečno manjkrat ocenjene kot uprabniki, poleg tega pa so po številu rešenih kvizov uporabniki precej neenakomerni.
%TODO: figure reševanja nalog
\subsection{Glicko rating na pridobljenih podatkih}
%label?

\iffalse
Če študent reševal skozi semester izgleda rating po Elo tako: 

(Če pa nimamo rednih podatkov):
zgenerirana tabela resevanja po kolokvijih, sprothin kvizih in random resevanjih. Na random ratingih nalog zajetih iz equiz ali
pa ce je kaj informacije o tem.

iz zgornjih tabel se da ugotavljati kasnejse ugotovitve napisane spodaj

(morda) imamo informacijo samo o izpitu.
(primerjava ratinga Elo vs Glicko na izpitu, tudi za naloge)

Če 
Trenutni podaki (lahko) baza za nove,

Po rating updatih lahko dolocimo cas (in verjetno kaj je ucenec reseval) -- match lahko uporabimo kar prvi rating in matchamo casovno naslednje preizkuse

Ali imamo podatke nakljucnih resevanj? (Nejc Subic diplomska: Da?)
Ce ne lahko tudi umetno prikazemo ucinek za sprotno delo -- nakljucno izberem nekaj nalog(lahko tudi randomly generiram podobne podatke kot trenutne naloge v equizu).

\begin{itemize}
    \item Prikaz ratingov po izpitu (tudi za naloge). kot poseben primer za prikaz da je se vedno accurate -- izstopanje nalog
    \item Prikaz ratingov po rednih preverjanjih, kot prikaz kako se vzdrzuje rating in devianca(graf deviance za studenta) -- c samo pokaze koliko globoko bo dipnilo med preverjanji in s tem koliko pomembno je sprotno delo?
\end{itemize}

Zeljeni podatki:
\begin{itemize}
    \item izpit (sasa)
    \item cas posodobitev ratingov + kaj je ucenec takrat reseval (za sestavljene naloge (izpite) + nakljucno resevanje)
\end{itemize}
\fi
S tem pridobimo v primeru študentov informacijo o sprotnem delu, saj bo RD manjši za učence, ki so pogosteje ocenjeni (poleg opravljenih sprotnih preverjanj še naključna reševanja kvizov/nalog), kar kaže na več/manj sprotnega dela.
(Glede na spreminjanje RD lahko spremljamo delo učenca kot nekakšen koeficient napredka/ak.vnos.)
Za nasprotnika učencu – nalogo, pa deviacija predstavlja kdaj je bila naloga nazadnje ocenjena, torej koliko je njena ocena zanesljiva. V primeru učenca, ki izbira naloge iz zbirke na equizu, pa lahko naloge z veliko deviacijo iden.ficiramo kot nepriljubljene (malo učencev se lo. naloge).
Reciklirane naloge bodo torej tudi večkrat nastopile na preverjanjih in bo natančnost njenih ra.ngov večja (dobro za parametrizirane naloge, ker se zaradi njihove narave lahko večkrat pojavijo v preverjanjih)
Ra.ng nalog, ki se pojavijo na izpitu, pa bo še vedno zanesljiv, saj bo v is. iteraciji nalogo reševalo veliko učencev.
Ra.ng učenca/naloge, ki je neak.ven čez določen časovni interval se ustrezno zmanjša, kar je poleg maksimalne deviacije edini parameter, ki ga je potrebno nastavi. kot administrator. Kako sistema.čno določamo ta parameter je opisano v dodatnem članku o glicko, vendar ga lahko verjetno vežemo na redna preverjanja kot npr. kolokvije. Ker se da predmet opravi. samo s kolokviji celotno snov sta.s.ke razdelijo na 5 delov. Torej bo študent, ki opravi samo eno preverjanje od pe.h ustrezno imel manjši ra.ng predvsem pa večjo deviacijo (če je opravljal samo prvega se bo njegov RD š.rikrat zviševal do konca semestra), saj imamo informacijo o njegovem znanju čedalje bolj nezanesljivo, ker vemo čedalje manj koliko študent dejansko ve o sta.s.ki odkar je reševal .s. kolokvij (pričakuje se, da študent tekom leta ve več in več o sta.s.ki).

%TODO tudi expected outcome

\hfill
\\
\\
\begin{lemma} \label{lem:eps}
$\ell = \varepsilon + t(p-1)(q-1)$.
\end{lemma}

\begin{proof}
\begin{align*}
\ell &= \varepsilon + \B{r}(q-1) \\
&= \varepsilon + t(p-1)(q-1)
\end{align*}
\end{proof}

\begin{theorem} \label{thm:equiv}
$\ell \equiv \varepsilon \pmod{\varphi(n)}$.
\end{theorem}

\begin{proof}
By Lemma~\ref{lem:eps} and \eqref{eqn:phi} we have $\varepsilon \bmod{\varphi(n)} = \ell$.
\end{proof}

\begin{corollary} \label{cor:mult}
$\exists k \in \Z: \ell - \varepsilon = k \varphi(n)$.
\end{corollary}

\begin{proof}
Follows by Theorem~\ref{thm:equiv}.
\end{proof}

\section{Glicko za spremljanje študentov}


\subsection{Implementacija}

\section{Conclusion}
\label{sec:cnc}
conclusion

\bibliographystyle{babplain}
\bibliography{cite}

\end{document}
